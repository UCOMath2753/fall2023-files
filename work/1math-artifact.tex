

\documentclass[12pt]{article}
\usepackage{amsmath}
\usepackage{fancyhdr}
\usepackage{hyperref}
\usepackage{graphicx}

\usepackage[top=1in, bottom=0.8in,  left=1in, right=1in]{geometry}
%\usepackage[]{anyfontsize}

\lhead{MATH 2753 -- Tech Prof Math Stat}
\rhead{Fall 2023}
\chead[RE]{\quad{{\bf \#1}, Due: Sept. 14}}
\cfoot{}

\pagestyle{fancy}
\begin{document}
\paragraph{Directions.} You will submit two files for this assignment.  The files should be named in the form ``last-first-latex1.tex'' and ``last-first-latex1.pdf'' (you may have to unzip and rename files if you download from Overleaf).  Use \url{https://classroom.github.com/a/Mn6NvA3o} to create, upload, and monitor your assignment.  

\paragraph{Options.} You have options for this assignment, briefly outlined below.
\begin{enumerate}
\item Type an approximately one-page solution that demonstrates the calculation of the derivative of \(f(x)=3x^2-7x+5\) using the limit definition of the derivative. Show all important steps in aligned equations. Use appropriate mathematical notation. Verify your answer using the power rule and report that result using at least one sentence that includes mathematical statements.
\item Write up a comparably difficult homework problem or class example from a current or recent class.  You should choose something that uses a mix of text and words and does not contain graphs or tables. Just math and at least one aligned calculation.
\end{enumerate}
Additionally, 
\begin{itemize}
\item make a title page that contains your name, the assignment title ``\LaTeX \#1: a mathematical artifact'', and the date the assignment is due.
\item use section and subsection headings, if relevant, but do not include a table of contents.
\item after \verb|\end{document}|, in a commented section called ``Bug report/Feature request'', make notes on any challenges you faced (e.g., errors or error messages) or request any special features you would like to incorporate in this or other assignments (e.g., though we've done this ``How to I emphasize something with color?'').
\item \emph{if you use any commands that have not been introduced in class, you should describe these (and your need for them) as comments within your \texttt{.tex} file.}
\end{itemize}


\end{document}