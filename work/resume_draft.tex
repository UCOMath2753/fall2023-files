\documentclass[11pt]{article}
\usepackage{amsmath, amssymb}
\usepackage{fancyhdr}
\usepackage{hyperref}
\usepackage{graphicx}
\newcommand{\compactlist}{\setlength{\itemsep}{0pt} \setlength{\parskip}{0pt} \setlength{\leftskip}{-1em}}

\usepackage[top=0.8in, bottom=0.8in,  left=1in, right=1in]{geometry}
\usepackage{pdfpages}
\lhead{MATH 2753 -- Tech Prof Math Stat}
\rhead{Fall 2021}
\chead[RE]{\quad\textbf{Project: Resume}}
\cfoot{}

\pagestyle{fancy}
\begin{document}
\paragraph{Overview.} Please use ``last-first-resume-draft.tex'' and ``last-first-resume-draft.pdf'' for your filenames.  Submit using the GitHub link accompanying this file on D2L by \textbf{\underline{{Tuesday, October 10}}}.

\paragraph{Background.} A r\'esum\'e or CV is an important document when you enter the job market.  A r\'esum\'e is traditionally used outside of the academic environment, while a CV is expected in most academic positions and tends to be a bit longer.  \textbf{See the worksheet at \url{https://bit.ly/3EBErOD} for additional tips and examples.} They overlap in much of the information that they provide:
%
\begin{itemize} \compactlist
\item Name and basic biographical\slash contact information -- \textbf{you can use fake/placeholder information, for example, the UCO campus mailing address as your contact information}
\item Education -- institution, degrees, years
\item Work History\slash Professional experience -- position, location, dates, duties for relevant employment or volunteer activities
\item Honors and Awards
\item Skills (e.g., software, special equipment)
\item References -- `letter writers' with contact information, a separate page following a r\'esum\'e
\end{itemize}
%
From here the documents might vary a bit in terms of emphasis.  Please consult the UCO guide for additional information on a resume, paying close attention to your intended profession where possible.

\paragraph{Commands.} You might\slash should 
\begin{itemize}\compactlist
\item incorporate some of the information in a nice header
\item use \verb|description| to format -- think of section names for \textbf{Work History}
\item use \verb|tabular| to format information -- think of formatting \textbf{Contact Information}
\item use \verb|itemize| to list information under sections
\item use ``\verb|As of: \today|'' in the header to display the date the document was prepared
\end{itemize}

\paragraph{Template.} Start with a blank document.  Get your information onto the screen \textit{then} focus on formatting.  Keep it simple, but interesting, for the first draft.   There are templates online, but you should develop your own document for this assignment.  You should be able to explain every single line of the code to me.

\paragraph{Submission.} The first attempt is due \textbf{Oct~10, 2022}.  A resubmission will account for some portion of your final exam grade.  Treat this assignment seriously -- this is equivalent to the Chapter 1 test for this   class.  \textbf{Do not forget to include a commented out ``Feature request/Bug report'' at the conclusion of the .tex file or on your GitHub submission and to upload the .pdf file.}  

\end{document}