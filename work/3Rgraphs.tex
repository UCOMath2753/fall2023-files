\documentclass[12pt]{article}
\usepackage{amsmath}
\usepackage{fancyhdr}
\usepackage{hyperref}
\usepackage[margin=1in]{geometry}

\lhead{MATH 2753 -- Tech for Prof Math Stat}
\rhead{Fall 2023 (11922)}
\chead[RE]{\qquad \bf{Prompt \#3}}
\cfoot{}

\newcommand{\compactlist}{\setlength{\itemsep}{0pt} \setlength{\parskip}{0pt} \setlength{\leftskip}{-1em}}


\pagestyle{fancy}

\begin{document}
\begin{description}
\item[Goal] Combine {\LaTeX} and R.
\item[Needs] Submit the following files
\begin{itemize}
\item \texttt{last-first-graph.R} - the script to generate a graph
\item \texttt{last-first-graph.pdf} - the graph
\item \texttt{last-first-graph-limit.tex} - the {\LaTeX} document
\item \texttt{last-first-graph-limit.pdf} - the pdf generated by the {\LaTeX} document
\end{itemize}
insert \emph{your} last and first names in the relevant positions in the filenames. These will be submitted to GitHub (\url{https://classroom.github.com/a/hY1ErHLQ}) by Thursday, Nov. 2 (negotiable).
\item[Content] You will generate a graph in R and embed it in a {\LaTeX} document that provides some useful context.  For example,  you might want to explore \verb|\begin{cases}...\end{cases}| as a way to define the piecewise function in the context of the document.
\begin{enumerate}
\item Build a graph of a piecewise function in R, with at least two pieces and open and closed circles to indicate where the function is defined at any point(s) of discontinuity.  Save the graph as a pdf. So you have control over the plotting window, it might be useful so start with
\begin{verbatim}
plot(NULL, xlim = c(#, #), ylim = c(#, #), ...)
\end{verbatim}
the references to \verb|#| and \verb|...| allow you to choose the limits accordingly and add any plot options.  To add your curves use
\begin{verbatim}
plot(function(x)..., xlim = c(#, #), ..., add=T)
\end{verbatim}
where again the \verb|#| offer placeholders for entering endpoints of the interval where that piece of the function is defined and the \verb|...| gives a place to enter the function formula (first) or any additional plotting comments (second).
\item Embed this graph in a {\LaTeX} document with a simple title page. \emph{Using your graph} state three relevant calculus limit problems and give their solutions.
\end{enumerate}
There is a good chance that the content fits on a single page.  Ask any questions you have about this in class, in office hours, or by email.

\end{description}
\end{document}