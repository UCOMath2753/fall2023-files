\documentclass[12pt]{article}

\usepackage{geometry}
\usepackage{amsmath}
\usepackage{amsthm}
\newtheorem{dfn}{Definition}[subsection]

\theoremstyle{definition}%% sample default aesthetic preferences
%% for more detail see
%% https://en.wikibooks.org/wiki/LaTeX/Theorems
%% symbols
%% https://www.cs.cmu.edu/~bhudson/symbols-letter.pdf
\newtheorem{thm}{Theorem}[subsection]
\renewcommand{\qedsymbol}{\emph{Q.E.D.}}

%% for more fun, check the comprehensive latex symbol guide for more options

\title{Day \#8: Mathematical environments}
\author{Sean Laverty}
\date{Monday, September 18, 2023}

\begin{document}
\maketitle
\newpage

\section{Mathematical Environments}
We have probably encountered math content with certain kinds of labels, for example,
\begin{itemize}
\item definition
\item theorem
\item proof
\item lemma
\item corollary
\item example
\item named equations, formulas
\item figure (but we've handled this separately, along with tables)
\end{itemize}

\section{Examples of Mathematical Environments}
\subsection{Examples of Definitions}
%% from preamble:
%% \newtheorem{definition}{Def.}[subsection]
%% the default is to italicize the contents, but you could change this
%% you could do this with more global defaults, or with more careful editing

\begin{dfn}[antiderivative]
A function \(F\) is called an \emph{antiderivative} of a function \(f\) on a given open interval if \(F'(x) = f(x)\) for \(x\) in the interval.
\end{dfn}

\begin{dfn}[your choice]
Write your own definition of a concept of \emph{your choice}.
\end{dfn}

\subsection{Examples of Theorems and Proofs}
%% from preamble:
%% \theoremstyle{plain}
%% \newtheorem{thm}{Theorem}[subsection]
In this unit, we discuss the rules for derivatives.  We begin with the \emph{power rule for derivatives} in Theorem~\ref{thm::power}.
%% use the built-in tracking system with \label{} and \ref{}

\begin{thm}[Power rule for derivatives]\label{thm::power}
If \(n\) is a positive integer, then \[\frac{d}{dx}\left(x^{n}\right) = nx^{n-1}.\]
\end{thm}
%% proof environment is built-in
\begin{proof}
If we had more time, we could prove this using a combination of the binomial theorem and the limit definition of the derivative.
\end{proof}


\end{document}
