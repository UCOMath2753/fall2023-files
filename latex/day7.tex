%% bit.ly/bibtex
%% search "bibtex wikibook" - https://en.wikibooks.org/wiki/LaTeX/Bibliography_Management

\documentclass[12pt]{article}

\usepackage{geometry}
\usepackage[round]{natbib}

\author{Sean Laverty}
\title{Day \#7: bibtex}
\date{Wednesday, September 13, 2023}

%% required for operatorname
\usepackage{amsmath} 
%% custom commands (macros)
\newcommand{\Var}[0]{\operatorname{Var}}
\newcommand{\ddx}[1]{\frac{d #1}{dx}}

\begin{document}
\maketitle
\newpage


I've spent all morning writing up notes for one of my other classes using PreTeXt, it's a language that sort of hybridizes {\LaTeX} and xml or html.  I also read about weird, possibly fake, topics in American history \citep{smit54}.  We have recently switched to an open-source textbook for our calculus sequence \citep{openstax}.
%% \citep{key} does parenthetical citation
%% \cite{key} and \citet{key} do in-text citations

In statistics we are often interested in variance, for a variable \(X\), we might be interested in the quantity \(\Var(X)\). A derivative is \(\ddx{f}\).
%% from preamble
%% \newcommand{\Var}{\operatorname{Var}}
%% \newcommand{\ddx}[1]{\frac{d #1}{dx}}

\bibliography{sample.bib}
\bibliographystyle{unsrtnat} %% requires natbib package

%% below are built-in and use \cite{}
%% options: plain, alpha, ... 
%% abbrv - first initial
%% apalike - full names + year
%% unsrt

%% more choices with a new package: natbib
\end{document}