\RequirePackage{pdfmanagement-testphase}
\DeclareDocumentMetadata{uncompress,pdfversion=2.0}

\documentclass[11pt]{article}
\usepackage{amsmath}
\usepackage{multirow}
\usepackage{xcolor}
\usepackage{graphicx}
\usepackage[margin=1in]{geometry}
\usepackage{tagpdf}
\tagpdfsetup{activate,paratagging,interwordspace}
\usepackage{fancyhdr}
\usepackage{hyperref}
\newcommand{\R}{\texttt{R}}

\newcommand{\compactlist}{\setlength{\itemsep}{0pt} \setlength{\parskip}{0pt} \setlength{\leftskip}{-1em}}
\renewcommand{\headrulewidth}{0.3pt}


\lhead{MATH 2753 -- TPMS}
\rhead{Fall 2023}
\chead[RE]{\hspace{2cm}\bf Math aRt}
\cfoot{}
\rfoot{\thepage}
\pagestyle{fancy}
\setlength{\parindent}{0pt}

\begin{document}
Work on the following alone or in groups today.
\begin{enumerate}
\item Sierpinski Triangle challenges
\begin{enumerate}
\item Rewrite the Sierpinski Triangle code in such a way that it adds a specifically-colored point to an existing plot each step of the calculation.
\item Modify that code to only plot the point if the step number is beyond a certain value.
\item Experiment with the calculations performed at each step to generate warped versions of the triangle that are visually interesting.
\item Experiment with the calculations by using matrices and vectors to perform the calculation at each step.
\end{enumerate}
\item Line art challenges\footnote{Visit \url{https://bit.ly/TPMS-math-art} for more context.} Use the following code template to generate digital ``string art''.
\begin{verbatim}
plot(NULL, xlim = c(-1, 2), ylim = c(-1, 2), xlab = "", ylab = "", axes=F)
n <- 100
for(i in 1:n){
	x0 <- cos(2*pi*(1 * i/n))
	x1 <- 1 - cos(2*pi*(1 * i/n))
	y0 <- sin(2*pi*(1 * i/n))
	y1 <- 1 - sin(2*pi*(1 * i/n))
	segments(x0, y0, x1, y1, lwd = 0.1)
}
\end{verbatim}
\end{enumerate}
\end{document}